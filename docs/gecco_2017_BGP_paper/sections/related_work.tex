\section{Related Work (one column)}
\label{sec:related-work}

\TODO{Talk about BGP in the context of program synthesis in general:  Armando, Gulwani, very recent Deep learning stuff!}

\TODO{Is any of the material below from another source? I wouldn't have thought of Memetic GP! }
One extension to the genetic programming paradigm is behavioral
genetic programming (BGP) \cite{krawiecGECCO2014,KrawiecBPS2016}.  BGP attempts to identify
useful subprograms that can then be used to enhance the evolutionary
process. The behaviroal program syhnthesis with Genetic
Programming\cite{krawiec2016behavioral} is investigated, where the
goal is to automatically generate a program that meets some
requirements. E.g. BGP looks at implicit fitness
sharing\cite{mckay2000fitness}, trace consistency analysis, memory in
the form of archives\cite{haynes1997line}. Extension are e.g. Memetic Semantic Genetic
Programming\cite{Ffrancon:2015:MSG:2739480.2754697} compute the
optimal ``shouldbe'' values each subtree should return.

\TODO{We should look at who cites the GECCO 2014 and book. That may point to related work.}

\TODO{ Also I looked through KK's papers since 2014 and the ones where he and Liskowski use NN-matrix factorization could be included. I added them in the .bib  }  Since its inception\cite{KrawiecSwan} and the introduction of the trace matrix and demonstration of blah in a system called IFORGET where X was the point, there have been a variety of extensions. For example,  in the general vein of behaviorally characterizing a program by more than its accuracy, K and Liskowski investigate which test cases it solves..


Semantic GP, where program output is scrutinized for every test
individually, initiated by a study of the impact of crossover on
program semantics and semantic building
blocks~\cite{mcphee2008semantic}. A survey of semantic models in
GP~\cite{vanneschi2014survey} where semantics is intended as the set
of output values of a program on the training data. This paper gives
an overview of methods for initialization, variation operators, as
well as thedifferent objectives of perserving of semantic diversity,
locality and geometry of topological spaces.

