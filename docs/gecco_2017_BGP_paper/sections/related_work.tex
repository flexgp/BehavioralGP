\section{Related Work (one column)}
\label{sec:related-work}

BGP is not the only approach to program synthesis, there are e.g. sketching\cite{solar2008program}, i.e. communicating insight through a partial program, generalized program verification\cite{srivastava2010program}, as well as hybrid computing with neural network and external menory\cite{graves2016hybrid}

One extension to the genetic programming paradigm is behavioral
genetic programming (BGP) \cite{krawiecGecco2014,KrawiecBPS2016}.  BGP
attempts to identify useful subprograms that can then be used to
enhance the evolutionary process. The behaviroal program syhnthesis
with Genetic Programming\cite{krawiec2016behavioral} is investigated,
where the goal is to automatically generate a program that meets some
requirements. E.g. BGP looks at implicit fitness
sharing\cite{mckay2000fitness}, trace consistency analysis, memory in
the form of archives\cite{haynes1997line}. Extension of BGP are
e.g. Memetic Semantic Genetic
Programming\cite{Ffrancon:2015:MSG:2739480.2754697}.

Since its inception\cite{krawiec2013pattern} and the introduction of
the trace matrix and demonstration in a system called PANGEA where
minimum description length was induced from the program execution,
there have been a variety of extensions. For example, in the general
vein of behaviorally characterizing a program by more than its
accuracy, e.g. when discovering search objectives for test-based
problems\cite{liskowski2016online}.

The relationship between BGP and Semantic GP is the common goal of
characterising program behaviour in more detail, and exploit this
information. Wheras BGP looks at subprograms semantic GP, where
program output is scrutinized for every test individually, was
initiated by a study of the impact of crossover on program semantics
and semantic building blocks~\cite{mcphee2008semantic}. A survey of
semantic models in GP~\cite{vanneschi2014survey} gives an overview of
methods for operators, as well as different objectives.

