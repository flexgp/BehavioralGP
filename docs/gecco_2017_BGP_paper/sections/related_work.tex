\section{Related Work}
\label{sec:related-work}

One extension to the genetic programming paradigm is behavioral
genetic programming (BGP) \cite{krawiec}.  BGP attempts to identify
useful subprograms that can then be used to enhance the evolutionary
process. The behaviroal program syhnthesis with Genetic
Programming\cite{krawiec2016behavioral} is investigated, where the
goal is to automatically generate a program that meets some
requirements. E.g. BGP looks at implicit fitness
sharing\cite{mckay2000fitness}, trace consistency analysis, memory in
the form of archives\cite{haynes1997line}. Extension are e.g. Memetic Semantic Genetic
Programming\cite{Ffrancon:2015:MSG:2739480.2754697} compute the
optimal ``shouldbe'' values each subtree should return.


Semantic GP, where program output is scrutinized for every test
individually, initiated by a study of the impact of crossover on
program semantics and semantic building
blocks~\cite{mcphee2008semantic}. A survey of semantic models in
GP~\cite{vanneschi2014survey} where semantics is intended as the set
of output values of a program on the training data. This paper gives
an overview of methods for initialization, variation operators, as
well as thedifferent objectives of perserving of semantic diversity,
locality and geometry of topological spaces.

