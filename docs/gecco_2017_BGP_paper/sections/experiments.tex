
\section{Experiments}\label{sect:experiments}

\subsection{Experimental Data, Parameters}\label{sect:data_sets}

All of the data sets used are defined such that the dependent variable is the output of a particular mathematical function for a given set of inputs.  They are taken from a paper entitled \textit{Genetic Programming Needs Better Benchmarks} by McDermott et al. \cite{benchmarks}  All of the inputs are taken to form a grid on some interval.  Let $E[a, b, c]$ denote $c$ samples equally spaced in the interval $[a,b]$. (Note that McDermott et al. defines $E[a, b, c]$ slightly differently.)  Below is a list of all of the data sets that are used:

\begin{enumerate}[noitemsep]
\item \textbf{Keijzer1}: $0.3x \sin(2 \pi x);$ $x \in E[-1,1,20]$
\item \textbf{Keijzer11}: $x y+\sin((x-1)(y-1));$ $x, y \in E[-3,3,5]$
\item \textbf{Keijzer12}: $x^{4}-x^{3}+\frac{y^{2}}{2}-y;$ $x, y \in E[-3,3,5]$
\item \textbf{Keijzer13}: $6 \sin(x) \cos(y);$ $x, y \in E[-3,3,5]$
\item \textbf{Keijzer14}: $\frac{8}{2 + x^{2} + y^{2}};$ $x,y \in E[-3,3,5]$
\item \textbf{Keijzer15}: $\frac{x^{3}}{5} - \frac{y^{3}}{2} - y - x;$ $x, y \in E[-3,3,5]$
\item \textbf{Keijzer4}: $x^{3} e^{-x} \cos(x) \sin(x) (\sin^{2}(x) \cos(x) - 1);$ $x \in E[0,10,20]$
\item \textbf{Keijzer5}: $\frac{3 x z}{(x - 10) y^{2}};$ $x,y \in E[-1,1,4]; z \in E[1,2,4]$
\item \textbf{Nguyen10}: $2 \sin(x) \cos(y);$ $x,y \in E[0,1,5]$
\item \textbf{Nguyen12}: $x^{4} - x^{3} + \frac{y^{2}}{2} - y;$ $x,y \in E[0,1,5]$
\item \textbf{Nguyen3}: $x^{5} + x^{4} + x^{3} + x^{2} + x;$ $x \in E[-1,1,20]$
\item \textbf{Nguyen4}: $x^{6} + x^{5} + x^{4} + x^{3} + x^{2} + x;$ $x \in E[-1,1,20]$
\item \textbf{Nguyen5}: $\sin(x^{2}) \cos(x) - 1;$ $x \in E[-1,1,20]$
\item \textbf{Nguyen6}: $\sin(x) + \sin(x + x^{2});$ $x \in E[-1,1,20]$
\item \textbf{Nguyen7}: $\ln(x + 1) + \ln(x^{2} + 1);$ $x \in E[0,2,20]$
\item \textbf{Nguyen9}: $\sin(x) + \sin(y^{2});$ $x,y \in E[0,1,5]$
\item \textbf{Sext}: $x^{6} - 2 x^{4} + x^{2};$ $x \in E[-1,1,20]$
\end{enumerate}


\textbf{Fixed Parameters}\label{appendix:fixed_parameters}

\begin{itemize}
\item \textbf{Tournament size}: 4
\item \textbf{Population size}: 100
\item \textbf{Number of Generations}: 250
\item \textbf{Maximum Program Tree Depth}: 17
\item \textbf{Function set}: $\{ +, -, *, /, \log, \exp, \sin, \cos, -x \}$
\item \textbf{Terminal set}: Only the features in the data set.
\item \textbf{Archive Capacity}: 50
\item \textbf{Mutation Rate $\mu$}: 0.1
\item \textbf{Crossover Rate with Archive configuration $\chi$}: 0.0
\item \textbf{Crossover Rate with GP $\chi$}: 0.9
\item \textbf{Archive-Based Crossover Rate $\alpha$}: 0.9
\item REPTREE  algorithm parameters 
\item Scikit learn algorithm parameters
\end{itemize}

We are investigating with 17 symbolic regression benchmarks.\\


First we replicated with REPTREE KK et al's work on the symbolic regression benchmarks. \\
Explain meaning of BGP2A, BGP4, BGP4A\\
Our open source software is available on GIT.\\


\subsection{Feature Selection Sensitivity}\label{sect:ftr-select}


Q1. Does the feature selection bias of the model step matter? 

Describe difference in implementations between Reptree and SKL-RepTree.\\
Need REPTREE and Scikit learn algorithm references and links to their code.

Compare REPTREE TO SICKIT LEARN for BGP 2A, 4a, 4 getting 6 combinations\\
Reference Table~\ref{table:avg_fitness} comparing average program error among BGP2A, BGP4, BGP4A, GP for 17 functions with REPTREE and ScikitLearn implementation.\\
Reference Table~\ref{table:avg_size} showing average program size among BGP2A, BGP4, BGP4A, GP for 17 functions with REPTREE and ScikitLearn implementation.\\
STDEV is in separate table in thesis, how to handle in paper?

Statistical testing required!!


\begin{table*}[ht]
\centering
\begin{adjustbox}{width=1\textwidth}
\small
\begin{tabular}{ c c c c c c c c c c c c c c c c c c c }
\hline\hline
 & & Keij1 & Keij11 & Keij12 & Keij13 & Keij14 & Keij15 & Keij4 & Keij5 & Nguy10 & Nguy12 & Nguy3 & Nguy4 & Nguy5 & Nguy6 & Nguy7 & Nguy9 & Sext \\
 \hline
GP &  & 0.303 & 0.851 & 0.968 & 0.391 & 0.841 & 0.879 & 0.576 & 0.986 & 0.163 & 0.381 & 0.221 & 0.246 & 0.128 & \textbf{0.004} & \textbf{0.077} & 0.108 & 0.076 \\
\hline
BP2A & REPTree & 0.272 & 0.784 & 0.97 & \textbf{0.306} & 0.887 & 0.874 & \textbf{0.344} & 0.974 & 0.13 & 0.352 & 0.225 & 0.247 & 0.033 & 0.101 & 0.119 & 0.069 & 0.054 \\
& SCIKitLearn & & & & & & & & & & & & & & & & & \\
 \hline
BP4 & REPTree & 0.467 & 0.908 & 0.989 & 0.914 & 0.916 & 0.915 & 0.706 & 0.997 & 0.412 & 0.444 & 0.222 & 0.327 & 0.275 & 0.086 & 0.113 & 0.201 & 0.26 \\
 %& Lasso & 0.4 & 0.818 & 0.989 & 0.742 & 0.863 & 0.959 & 0.573 & 0.989 & \textbf{0.09} & 0.431 & 0.434 & 0.558 & 0.171 & 0.243 & 0.284 & 0.184 & 0.18 \\
 & Scikit Learn & 0.32 & \textbf{0.547} & \textbf{0.95} & 0.43 & 0.874 & 0.885 & 0.475 & 0.989 & 0.098 & 0.387 & 0.244 & 0.269 & 0.111 & 0.111 & 0.113 & 0.082 & 0.083 \\
% & Randomized & 0.468 & 0.855 & 0.989 & 0.902 & 0.92 & 0.919 & 0.694 & 0.997 & 0.428 & 0.437 & 0.253 & 0.366 & 0.253 & 0.073 & 0.139 & 0.186 & 0.201 \\
 \hline
BP4A & REPTree & 0.435 & 0.912 & 0.988 & 0.891 & 0.92 & 0.909 & 0.672 & 0.997 & 0.37 & 0.438 & 0.206 & 0.352 & 0.152 & 0.154 & 0.132 & 0.221 & 0.217 \\
% & Scikit Learn & 0.357 & 0.701 & 0.975 & 0.46 & 0.801 & 0.913 & 0.397 & 0.982 & 0.274 & \textbf{0.347} & 0.226 & 0.329 & 0.038 & 0.027 & 0.115 & 0.076 & 0.076 \\
% & Randomized & 0.471 & 0.92 & 0.989 & 0.874 & 0.923 & 0.938 & 0.706 & 0.997 & 0.404 & 0.432 & 0.239 & 0.334 & 0.201 & 0.119 & 0.105 & 0.196 & 0.154 \\
% & Larger Archive & 0.443 & 0.919 & 0.988 & 0.897 & 0.917 & 0.936 & 0.69 & 0.997 & 0.356 & 0.421 & 0.206 & 0.306 & 0.167 & 0.179 & 0.131 & 0.183 & 0.22 \\
% & Different Rates & 0.436 & 0.909 & 0.988 & 0.873 & 0.916 & 0.918 & 0.681 & 0.997 & 0.4 & 0.413 & \textbf{0.186} & 0.375 & 0.135 & 0.123 & 0.127 & 0.303 & 0.212 \\
& SCIKitLearn & & & & & & & & & & & & & & & & & \\
\hline
\end{tabular}
\end{adjustbox}
\caption{Average program error for best of run programs.}
\label{table:avg_fitness}
\end{table*}

\begin{table*}[ht]
\centering
\begin{adjustbox}{width=1\textwidth}
\small
\begin{tabular}{ c c c c c c c c c c c c c c c c c c c }
\hline\hline
 & & Keij1 & Keij11 & Keij12 & Keij13 & Keij14 & Keij15 & Keij4 & Keij5 & Nguy10 & Nguy12 & Nguy3 & Nguy4 & Nguy5 & Nguy6 & Nguy7 & Nguy9 & Sext \\
 \hline
GP &  & 38.33 & 43.57 & 48.97 & 28.1 & 29.1 & 45.77 & 58.1 & 42.83 & 21.2 & 23.47 & 24.7 & 29.9 & 23.67 & 10.67 & 28.63 & 23.53 & 32.97 \\
\hline
BP2A & REPTree & 37.6 & 48.87 & 65.57 & 38.97 & 28.03 & 56.2 & 71.17 & 44.9 & 27.83 & 40.13 & 42.6 & 44.63 & 22.17 & 25.63 & 31.57 & 23.53 & 37.1 \\
 & Full Pop & 67.03 & 67.23 & 119.8 & 52.6 & 48.33 & 120.87 & 92.57 & 71.5 & 57.5 & 56.13 & 64.97 & 61.87 & 32.37 & 44.4 & 51.83 & 47.9 & 65.47 \\
 & Scikit Learn & 41.6 & 44.2 & 40.6 & 24.8 & 43.4 & 74.2 & 63.6 & \textbf{25.2} & 30.6 & 31.6 & 35.8 & 59.6 & 31.4 & 16.6 & 38.0 & 14.2 & 31.0 \\
 & Randomized & 41.77 & 40.13 & 51.53 & 43.97 & 29.63 & 59.7 & 49.7 & 44.57 & 36.27 & 34.0 & 43.17 & 37.2 & 25.17 & 21.83 & 29.5 & 14.13 & 43.03 \\
 & Larger Archive & 38.63 & 38.97 & 66.97 & 43.97 & 26.33 & 57.47 & 75.23 & 44.47 & 33.3 & 38.3 & 32.7 & 40.9 & 21.13 & 22.57 & 31.6 & 31.53 & 43.9 \\
 & Different Rates & 36.53 & 38.33 & 55.97 & 43.0 & 24.97 & 58.13 & 48.13 & 48.77 & 32.77 & 38.0 & 32.47 & 41.93 & 25.9 & 26.67 & 34.33 & 22.8 & 35.27 \\
 \hline
BP4 & REPTree & \textbf{20.27} & \textbf{20.8} & \textbf{27.37} & \textbf{24.47} & \textbf{17.9} & 44.47 & \textbf{22.3} & 51.2 & 17.4 & \textbf{18.97} & 24.93 & 25.2 & 14.87 & 20.5 & 29.47 & 12.17 & 22.87 \\
 & Lasso & 20.87 & 29.13 & 27.93 & 25.9 & 26.9 & \textbf{31.33} & 33.27 & 64.43 & 15.27 & 27.4 & \textbf{22.77} & 20.03 & 17.3 & 13.13 & \textbf{19.27} & 12.13 & \textbf{20.1} \\
 & Scikit Learn & 37.6 & 29.6 & 44.0 & 25.2 & 29.2 & 33.4 & 38.2 & 47.8 & \textbf{13.0} & 26.8 & 25.0 & \textbf{20.0} & 23.0 & \textbf{10.6} & 55.2 & 17.2 & 25.4 \\
% & Randomized & 28.43 & 21.17 & 29.83 & 25.0 & 18.77 & 46.07 & 22.87 & 40.27 & 14.5 & 19.9 & 28.47 & 28.23 & 14.1 & 17.1 & 26.67 & \textbf{9.1} & 21.73 \\
 \hline
BP4A & REPTree & 34.8 & 29.03 & 49.67 & 28.2 & 23.23 & 47.47 & 32.6 & 53.07 & 18.37 & 23.27 & 33.87 & 35.83 & 13.97 & 39.23 & 32.07 & 16.2 & 28.23 \\
 & Scikit Learn & 26.6 & 36.0 & 62.2 & 30.0 & 36.4 & 65.2 & 51.0 & 42.6 & 24.0 & 39.6 & 42.8 & 64.6 & 27.0 & 13.2 & 35.4 & 10.6 & 40.4 \\
 & Randomized & 48.77 & 24.53 & 39.73 & 28.63 & 25.17 & 44.3 & 42.83 & 50.37 & 20.57 & 24.87 & 33.13 & 34.5 & 16.8 & 31.33 & 30.27 & 11.47 & 22.0 \\
 & Larger Archive & 42.4 & 35.37 & 43.17 & 37.03 & 22.17 & 51.8 & 39.33 & 51.03 & 17.1 & 25.4 & 34.63 & 36.03 & \textbf{13.73} & 30.87 & 34.63 & 15.17 & 24.9 \\
 & Different Rates & 38.43 & 28.5 & 41.23 & 29.33 & 25.07 & 52.07 & 36.93 & 56.93 & 19.27 & 27.2 & 33.17 & 37.5 & 15.33 & 34.1 & 29.77 & 15.23 & 34.5 \\
\hline
\end{tabular}
\end{adjustbox}
\caption{Average program size for best of run programs.}
\label{table:avg_size}
\end{table*}

Include a ranking table.  just on error of program and only 6 combinations (two choices of decision tree implementation) and 3 algorithms (2A,4,4A). 

Select 1 run, 1 dataset: what is \# of features in model of best individual at first and final generation (each generation)? What is fraction of \# of features in model to \# of subtrees in best individual at first and final generation (each generation)?  (Each generation) means we would have a plot (fitness on Y1 axis, features/fraction on Y2, Y3 and X is generations). We could amass these statistics for every dataset and 30 runs but lower priority than other tasks.

Segue to next subsection: Do features depend on one another? We're identifying their model value in the context of the same tree. They could be co-adapted and not very good with any other tree. We investigate further.

\subsection{Aggregate Trace Matrices}\label{sect:agg-features}

The question is what aggregations? What if we take everyone?

show discuss results of full-pop (call it 100_XX)\\

Detailed study: take 1 run, 1 dataset (save seed!) and one algorithm of 2A,3,3A and one choice of decision tree.  What dataset? Choices: use hardest problem, or one that 2A,3,3A don't do as well as GP on. \\
How many programs have one or more features in the model?\\ how many programs have 2, 3, etc? This says something about co-adaptation. Were pairs of features subtrees where one was within the other?  \\what is the ratio of model features to total-features-in-pop? \\how many features are in the model (first, final generations?) (maybe plot every generation ( very low priority))

Now what if we discriminate the aggregation set by fitness so we're only identifying features from superior parts of the population?

We create 3 additional  aggregate trace matrices: 25, 50, 75,  where pop is sorted low to high by fitness (error + size) and then cut off at these different thresholds. 

Show/describe a longitudinal comparison of all 4 and add in best of program-trace resutls.

Does the discrimination by program fitness plus the aggregation among programs work better than just program trace? SHOW GRAND RANKING.

\TODO{Insert my sketch of tables of results}

\TODO{Not for Monday night but if we get accepted, we can increase the number of runs and robustness of the detailed results to encompass more datasets or more runs and provide average and standard deviation}
