\chapter{Introduction}
\label{chap:intro}
Genetic programming (GP) \cite{koza} is a subfield of Artificial Intelligence, in which the principles of evolution are algorithmically translated in order to produce programs with desired functionality.  Each such program is defined by a set of genes, which can be mutated and swapped in a manner similar to reproduction in biological organisms.  A typical genetic programming algorithm will start with a population of initial programs, which are randomly generated.  Each successive iteration, a population of new programs is generated from the old population.  Finally, the best programs from the two populations are selected, and the process is repeated until a sufficiently good program is found.

The power of this technique is that the programmer does not need to guess at the structure of the desired program.  All the programmer needs is a means of constructing new programs from old programs, and a fitness function by which to compare the performance of one program to another.  On the surface, this seems like a very promising method to produce arbitrarily complex programs, as long as the programmer has an understanding of the desired output.  However, in practice, the landscape of programs is enormous, and often times there are many local optima, which prevent evolving programs from achieving the desired functionality.

There are a variety of techniques and adaptations to the basic genetic programming methodology that attempt to exploit features of the evolutionary process to arrive at optimal solutions more quickly.  One such technique, designed by Krawiec et al. \cite{krawiec} attempts to utilize information about how to identify useful subprograms.  These subprograms are components of programs in the population that will help drive the evolutionary process, even if they are a part of a program that may not perform well on the specified task.  This technique is termed \textit{behavioral genetic programming} (BGP).  The vision of this work is that BGP is a paradigm rich with possible extensions to explore, many of which could give deeper insight into genetic programming methods.

This work focuses on replicating the results first achieved by Krawiec et al., and exploring various extensions to the BGP paradigm.  This work proceeds as follows. Chapter \ref{chap:relatedwork} discusses the basics of genetic programming, and the key ideas behind BGP.  Chapter \ref{chap:implementation} discusses my implementation of BGP with the added ability of running many different configurations that were not explored by Krawiec et al.  Chapter \ref{chap:experiments} discusses the experiments that are performed, and Chapter \ref{chap:conclusion} discusses my contributions and possible paths for future work.
