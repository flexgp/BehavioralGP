\section{Introduction}

Genetic Programming~(GP) has been applied to different problems,
e.g. in law there is a study of representing contracts as GP
trees~\citep{chandler2007genetically}. In addition, adversarial
coevolutionary algorithms have been used to model arms-races, e.g. in
cyber security~\citep{haddadi2015botnet}. We present an application of
GP on tax evasion, a fundamental and pervasive societal problem. Our
focus is on the domain of US Partnership taxation, which had a \$91
billion tax gap in 2012~\citep{gao2014}. Tax evasion can be seen as an
adversarial arms-race, the attacker is the tax evader, the defender is
the auditor. Tax auditors have historical examples of tax schemes to
help auditing. On the evasion side tax shelter promoters often adapt
their strategies as existing schemes are uncovered and when changes
are made to the existing tax regulations. One example is the so called
BOSS tax shelter (Bond and Options Sales Strategies) that was widely
promoted yet was ultimately disallowed. While audit changes were
implemented to detect BOSS they were not able to detect the newly
emerged variant ``Son of BOSS''~\citep{wright45financial}.

There are some technical challenges to modeling U.S. partnership taxation:
\begin{inparaenum}[\itshape A)]
\item the tax code's complexity
\item the behaviors available to tax evaders and auditors
\item the simultaneous co-adaptive behaviors of both auditors and tax
  evaders.
\end{inparaenum}
First, we need to develop an abstraction of the relevant partnership
tax law, which will allow us to compute both tax liability and
likelihood of being audited. Second, we have to develop a model of
taxpayer and auditor co-adaptive behavior. In the Simulation of Tax
Evasion and Law Through Heuristics~(STEALTH) framework, see
Figure~\ref{fig:STEALTH_overview}, we take knowledge from
Coevolutionary algorithms~\citep{popovici2012coevolutionary} to model
the dynamics of the coevolutionary adversarial relationship between
tax evader and auditor. Specifically the competitive coevolution with
two populations, i.e. interactive test-based coevolution. We chose a
method with an explicit use of grammar for search bias and to compress
the search space. Grammatical Evolution~(GE)~\citep{o2003grammatical}
is one such grammar based GP method. Grammars are appropriate start
point for defining, constraining and adding domain knowledge to the
search space of evader and auditor actions. In the experiments we
analyze the adversarial search dynamics of applying STEALTH to a tax
evasion scheme, called iBOB~\citep{iBOB}, designed to defer tax
payments by using US partnerships.

\begin{figure}
  \centering
  \includegraphics[width=0.99\linewidth]{figures/STEALTH_GPTP_STEALTH_overview}
  \caption{STEALTH Framework Overview}
  \label{fig:STEALTH_overview}
\end{figure}

In Section~\ref{sec:related-work} work on coevolution and GE are
described. The STEALTH framework is described in
Section~\ref{sec:method}. In Section~\ref{sec:experiments} there are
experiments regarding STEALTH. Finally, there are conclusions and
future work in Section~\ref{sec:conclusions--future}
