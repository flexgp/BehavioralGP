\section{Method}
\label{sec:method}

The STEALTH framework is composed of two modules:
\begin{inparadesc}
\item [regulatory module] the framework pertaining to US partnership
  taxation and another that simulates the adversarial dynamics that
  occur within such a framework.  The modules are shown in
  Figure~\ref{fig:STEALTH_details}. The legal logic to calculate tax
  liability and assign audit risk are here.
  \item [coevolutionary module] That module then contains the
  algorithms which replicate the desired dynamics. The values from the \textit{regulatory module} are passed to
  the individuals from which they were derived in the
  \textit{Coevolutionary Module}.
\end{inparadesc}
Next background will be provided on the regulatory module,
Section~\ref{sec:regulatorysystem}, to give context to the
coevolutionary module, Section~\ref{sec:optimization}.

\begin{figure}
  \centering
  \includegraphics[width=0.99\linewidth]{figures/STEALTH_GPTP_STEALTH_details}
  \caption{STEALTH Framework details}
  \label{fig:STEALTH_details}
\end{figure}

\subsection{Tax Regulatory Module}
\label{sec:regulatorysystem}

Tax consequences of financial activity must be computed in order to
analys abusive tax behavior. Therefore, many aspects of Subchapter K
of the IRC were hand coded into a succinct mathematical
representation. This representation needs the ability to process an
arbitrary set of financial behavior for tax liability. We settled on
the following notation, e.g.

{\small An asset is a tuple $(b, \beta, \tau)$ consisting of
\begin{inparaenum}[\itshape (1)]
  \item Adjusted Basis: A scalar $b \in \mathbb{R}^+$
  \item Book Value: A scalar $\beta \in \mathbb{R}^+$
  \item Type: A positive integer $\tau$ that whether the asset is
    category $0$ (cash), category $1$ (ordinary) and category $2$
    (capital).
\end{inparaenum}}

Once we are able to calculate the tax liability resulting from a
single transaction, we must determine what form a potentially abusive
tax strategy takes, and how to represent a given auditing policy. The
following sections describe our quantitative representations of tax
evaders and auditors.

\subsubsection{Tax Network and Transactions}
\label{sec:transactions}

We first define the environment of interest, namely the initial
conditions of a tax regulation unit. This amounts to a set of
entities, each of which have a "portfolio" of assets, the entirety of
which we refer to as the \textit{tax network}. In abstract terms, we
define the state of the network as some $\gamma \in \Gamma$, where
$\gamma = \{\textbf{e}, \textbf{a}, d\}$. The tuple $\gamma$ is
composed of a set of entities $\textbf{e} = \{e_i\}_{i=0}^{k_1}$, a
set of assets $\textbf{a} = \{a_i\}_{i=0}^{k_2}$ where $k_1,k_2 \in
\mathbb{Z}_+$. The operator $d$ determines the owner of each asset,
i.e~$d: A \mapsto E$, where $A$ is the space of assets and $E$ is the
space of entities.

Transactions between entities are tax network state transitions. A
transaction is considered a specific type of transition from one tax
network state $\gamma_n$ to another $\gamma_{n+1}$. A transaction is
thus described as some $t=\{e_f,e_t,a_f,a_t\}$, where $e_f,e_t \in E$
are two entities and $a_f,a_t \in A$ are two assets that are being
exchanged between the two entities. Finally we can define a
\textit{transaction sequence} as $\textbf{t} = \{t_i\}_{i=0}^k$, where
$k \in \mathbb{Z}_+$ is the number of transactions.

\subsubsection{Audit Score Sheets}
\label{sec:auditscoresheets}

We abstract a tax regulator for the tax regulatory model in order to
determine the likelihood of conducting an audit, similar to how a tax
regulated unit is represented as a transaction sequence. A regulator
is a certain auditing policy, which is represented as a list of events
\textit{observable} within a transaction sequence with numerical
weights associated with each type of event. When a particular
observable event is applied to the regulatory module, the overall
audit likelihood is incremented by its associated weight. The list of
observable events with weights is referred to as the \textit{audit
  score sheet}. The audit score sheet records the occurrence of each
type of observable event. But it can also record every possible
combination of the behaviors. For example, consider the following
passage from the Internal Revenue Code $\S 743(a)$.

{\tiny
\begin{quote}
  The basis of partnership property shall not be adjusted as the
  result of (1) a transfer of an interest in a partnership by sale or
  exchange or on the death of a partner unless (2) the election
  provided by $\S 754$ (relating to optional adjustment to a basis of
  partnership property) is in effect with respect to such partnership
  of (3) unless the partnership has a substantial built-in loss
  immediately after such transfer.
\end{quote}
}

Each number with parentheses signifies an \textit{observable}
event. Namely,
\begin{inparaenum}[(1)]
  \item The sale of a partnership interest in exchange for a
    \textit{taxable} asset.
  \item The partnership whose shares are being transferred has not
    made a \S 754 election.
  \item The seller's basis in respect to the non-cash assets owned by
    the partnership exceeds their FMV by more than $\$250,000$.
\end{inparaenum}
An audit score sheet that encapsulated only the three observable
events listed in the passage would look as follows.
\begin{table}[!htb]
  \small
  \centering
  \begin{tabular}{c|c|c}
    \textbf{Observable} & \textbf{Weights(w)} & \textbf{Frequency(f)} \\
    \hline
    Partnership Interest Sale ($1$) & $\textnormal{w}_1$ & $\textnormal{f}_1$ \\
    \hline
    No $\S 754$ Election ($2$) & $\textnormal{w}_{2}$ & $\textnormal{f}_{2}$ \\
    \hline
    Substantial built-in Loss ($3$) & $\textnormal{w}_{3}$ & $\textnormal{f}_{3}$ \\
    \hline
    $1 \cup 2$ & $\textnormal{w}_{1 \cup 2}$ & $\textnormal{f}_{1 \cup 2}$ \\
    \hline
    $1 \cup 3$ & $\textnormal{w}_{1 \cup 3}$ & $\textnormal{f}_{1 \cup 3}$ \\
    \hline
    $2 \cup 3$ & $\textnormal{w}_{2 \cup 3}$ & $\textnormal{f}_{2 \cup 3}$ \\
    \hline
    $1 \cup 2 \cup 3$ & $\textnormal{w}_{1 \cup 2 \cup 3}$ & $\textnormal{f}_{1 \cup 2 \cup 3}$ \\
  \end{tabular}
  \caption{{\small Each row has three columns with $1$) the type of
      observable corresponding to the three characterized observables
      $2$) the associated audit weight and $3$) the number of times it
      occurs in a list of transactions}}
  \label{tab:audit_score_sheet}
\end{table}

The formulation of the audit score sheet requires determing
observables from the tax regulations. In contrast the calculation of
the audit score sheet from is currently straightforward. Suppose that
there are $m$ specific types of events that are observable,
represented by $\{b_i\}_{i=0}^n$, where $n = 2^m - 1$, the entire
combinatoric space.. Associated with each type of event are the
weights $\{\alpha_i\}_{i=0}^n, \alpha \in \mathbb{R}_+$ and the
frequency that the event occurs within a network of transactions
$\{f_i\}_{i=0}^n, f_i \in \mathbb{Z}_+$. We can then write the audit
score, $s$ corresponding to the audit score sheet and network of
transactions as $s = \sum_{i=0}^n \alpha_i f_i \text{ where }
\sum_{i=0}^n \alpha_i = 1$

In total, the simulation is defined as a function $\textbf{F} :
\textbf{T} \times \Gamma \times \Phi \mapsto \mathbb{R}^{\zeta} \times
\mathbb{R}_+$, where $\textbf{T}$ is the space of all transaction
sequences, $\Gamma$ is the space of all initial tax networks, $\Phi$
is all audit score sheets, $\mathbb{R}^{\zeta}$ represents $\zeta$
measures of taxable income and $\mathbb{R}_+$ is the audit likelihood.

\subsection{Coevolutionary Module}
\label{sec:optimization}

We describe the module which directs the appropriate adversarial
dynamics. This module has two subtasks.
\begin{inparaenum}[\itshape A)]
\item assign \textit{fitness} to both transaction sequences and audit
  score sheets based on the measures of taxable income and audit
  score
\item co-adapt the two competing populations, of solutions and tests,
  in terms of the fitness score, by searching over its behavioral
  space.
\end{inparaenum}

Our adversarial multi population coevolutionary algorithm in STEALTH
is in the interactive test
domain\citep{popovici2012coevolutionary}. The interaction is between
populations of tax evasion schemes and audit policies. The tests are
respectively audit observables and tax evasion scheme. A potential
solution is a tax evasion scheme or audit observable weights.  The
problem is to find the least risky evasion scheme or most likely audit
observables. The domain is adversarial competition with the fitness of
the tax evader is the opposite of the auditor. The problem is of the
dual nature, i.e. the solutions for both sides are interesting.



\subsubsection{Adversarial Population Representation}
\label{sec:grammaticalevolution}

Individual solutions, i.e transaction sequences and audit score
sheets, are both evaluated in the tax regulatory system in separate
populations. Therefore we must express and explore the spaces of all
possible transaction sequences and audit weights. Grammatical
Evolution (GE) offers a solution.

We augment GE in a manner similar to GEGE to efficiently break up
problem and gradually increase the search space of entities in
STEALTH. Generating transaction sequences by removing some integers
from the vector, which is a method that we use to generate initial tax
network configurations. For example, if we would like to determine
some number of additional partnerships in the initial configuration
between $0$ and $k$, we remove the first integer and take its modulo
in respect to $k$, then process the rest of the vector through the
grammar.

Mapping an integer sequence to an audit score sheet, on the other
hand, is more straightforward due to the numerical qualities of the
audit weights. For an audit score sheet of length $m$ we simply take
an integer vector of length $m$ and divide each integer in the vector
by the sum of all of the elements. This creates $m$ positive real
numbers that sum to one.

\subsubsection{Coevlutionary Tests -- Objective Functions}
\label{sec:objectivefunction}

We take into account both tax liability and likelihood of being
detected by various auditing policies. Similarly, we search for an
effective auditing policy. Neither task is trivial, nor can they be
generalized to encompass every use case. We can apply a heuristic for
determining effectiveness in a specific scenario to help formulate a
good objective function. These heuristics are means to formulate
proper objective functions for both a tax-minimizing strategy and
audit weights, given a transaction sequence, initial tax network and
audit score sheet.

An \textit{objective function} is some mapping between the numerical
traits associated with a transaction sequence or audit score sheet,
and some measure of desirability. Section~\ref{sec:regulatorysystem}
describes how to calculate taxable income for all $\zeta$ of the
entities in the simulation, and an audit score. Given these two
numerical constructs, we formulate objective functions for both
transaction sequences, $h_e$, and audit score sheets, $h_s$, both are
defined as a mapping from measure of taxable income and audit
likelihood, to a real valued scalar.

An effective transaction sequence, from the perspective of a taxpayer,
results in a low level of taxable income, with a low likelihood of
being audited. A highly effective transaction sequence would be in the
lower left corner, incurring relatively low levels of tax liability
and little likeliehood. A transaction sequence that produces low
levels of tax liability but a \textit{high} likelihood of being
audited would be undesirable. We evaluate transaction sequences that
all accomplish relatively similar economic goals. Thus any lower
variations in taxable income can be indicative of, at the very least,
tax implications that were never intended by policy-makers.

Auditing policies face a different heuristic for calculating
effectiveness. Auditing policies must take into account the amount of
resources that it takes to audit. A good auditing policy avoid false
positives and negatives and applies a low audit likelihood to
transactions sequences that generated relatively normal levels of
taxable income and a high likelihood to similar, low taxable income
transaction sequences. Bad auditing policies are the exact opposite.

\subsubsection{Adaptation -- Coevolutionary Genetic Algorithm}
\label{sec:searching}

The next task is to specify a means by which a large and highly
non-linear space of transaction sequence-audit score sheet pairs can
be co-adapted. The objectives establish a notion of effectiveness, the
evolutionary algorithm determines
\begin{inparaenum}[\itshape a)]
\item which transaction sequences can minimize tax liability
  while circumventing an audit and
\item which auditing policies assign high audit likelihood to
  relatively low tax liability schemes while ignoring
  non-suspicious behavior.
\end{inparaenum}
The aim is to anticipate new forms of potentially abusive tax behavior
as well as desirable, or likely, regulator response to it.

Upon establishing these mappings, a search can be performed on the
space. Because of the predator-prey relationship between
non-compliance schemes and auditing policy, we chose to use a
\textit{co-evolutionary algorithm}. Specifically, we evolve a
test-based interaction problem. There are two competing populations of
soultions that evolve in paralel and the fitness of a solution is
subjective, i.e. the fitness dependes on the test that the solution
interacted with. Each individual in the two populations are evaluated
against a subset of the opposing population, which can be chosen by a
number of different decision heuristics.

Our coevolutionary algorithm:
\begin{asparaenum}[\itshape 1)]
\item \textbf{initializes} both populations
\item \textbf{evaluates} each individual against a subset of of the other
  population to determine their objective score
\item \textbf{selects} the best individuals in each population
\item creates new populations by \textbf{crossover}(combining) the
  chosen individuals
\item introduces slight \textbf{mutation} into that new population
\item \textbf{repeats} steps $2-5$ over some \textit{generations}
  until there is some halting condition.
\end{asparaenum}

Specifically, every generation, each individual in the transaction
sequence population selects a random subset of the audit score sheet
of the population to evaluate against. After all sequences are
evaluated, the process is repeated with the opposite population: each
audit score sheet chooses a random subset of the transaction sequence
population to evaluate. See~\cite{rosen2015thesis} for more details.

\subsubsection{Summary of coevolutionary module}

The regulatory system is a function $\textbf{F} : \textbf{T} \times
\Gamma \times \Psi \mapsto \mathbb{R}^{\zeta} \times \mathbb{R}_+$
that takes as input a sequence of transactions, an initial network
state and auditing observables, and generates the taxable income for
all relevant entities and audit score. In other words, for any
$\textbf{t} \in \textbf{T}$ and $\gamma_0 \in \Gamma$ generated from
the same vector of integers $\textbf{x}$ and accompanying auditing
observables $\psi \in \Psi$, $\textbf{F}\left(\textbf{t}, \gamma_0,
\psi \right) = \left({\bm \ell}, s \right)$, where ${\bm \ell}$ is a
vector of real numbers of length $\zeta$ that represents taxable
income for all entities and $s$ is the audit score.

The function $\textbf{F}$ can be broken up into a network of transition
functions that has the same length as the number of transactions in
the transaction set contained within the function call ($k$). Each
transition function generates a new network state and an audit score. So
for all $i \in [0,k]$, $F_i\left(t_i, \gamma_i, \psi \right) =
\left(\gamma_{i+1}, s_i \right) \quad \text{where} \quad s=s_k$


% TODO Jake, an example for clarification

The objective functions for transaction sequences and audit score
sheets are, respectively, $h_e$ and $h_s$, both maps from
$\mathbb{R}^{\zeta} \times \mathbb{R}_+$. Additionally, define
$\Xi_t:\mathbb{Z}_+^n \mapsto \textbf{T} \times \Gamma$ and
$\Xi_a:\mathbb{Z}_+^m \mapsto \mathbb{R}_+^m$ as, respectively, the
transaction sequence and audit score sheet mapping functions described
in Section~\ref{sec:grammaticalevolution} Now it is possible to fully
define the maximizing objectives of networks of transactions as

\begin{align*}
\arg\!\max_{\textbf{x}^* \in X} \left[
\text{\textit{h}}_e \left(\begin{matrix}
  \textbf{F}\left(\Xi_t(\textbf{x}^*), \Xi_a(\textbf{y}) \right)
 \end{matrix}\right)\right] &=
\arg\!\max_{\textbf{t}^* \in \textbf{T}, \gamma_0^* \in \Gamma}
\left[\text{\textit{h}}_e \left(\begin{matrix}
    \textbf{F}\left(\textbf{t}^*, \gamma_0^*, \psi \right)
\end{matrix}\right)\right] &
\end{align*}

over all $\textbf{y} \in B(\hat{\textbf{y}},r_1)$ for some
$\hat{\textbf{y}} \in \mathbb{Z}_+^m$, where $B(\hat{\textbf{y}},r_1)$
is a \textit{ball} of radius $r_1 \in \mathbb{R}_+$ around
$\hat{\textbf{y}}$. This represents the fact that the goal of the GA
is to find local maxima around some subset of auditing behavior,
rather than attempting to search the entire $\Phi$ space. Conversely,
the objective for the auditing behaviors is to maximize the
\textit{positive} $h_a$ function, the opposite of the objective for
the transactions, i.e. the goal is

\begin{align*}
\arg\!\max_{\textbf{y}^* \in \mathbb{Z}_+^m} \left[
\text{\textit{h}}_a \left(\begin{matrix} \textbf{F}\left(
  \Xi_t(\textbf{x}), \Xi_a(\textbf{y}^*)
  \right)\end{matrix}\right)\right] &= 
  \arg\!\max_{\psi^* \in \Psi}
\left[\text{\textit{h}}_a \left(\begin{matrix}
    \textbf{F}\left(\textbf{t}, \gamma_0, \psi^* \right)
\end{matrix}\right)\right]
\end{align*}

over all $\textbf{x} \in B(\hat{\textbf{x}},r_2)$ for some
$\hat{\textbf{x}} \in \hat{X}$, where $B(\hat{\textbf{x}},r_2)$ is a
\textit{ball} of radius $r_2 \in \mathbb{R}_+$ around
$\hat{\textbf{x}}$. Similar to the previous objective function, this
represents the fact that the EA only searches for local maxima around
a subset of all transaction sets and initial model states. Now we
investigate how STEALTH perofrms.
