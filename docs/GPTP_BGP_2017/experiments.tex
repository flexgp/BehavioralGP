\section{Experiments}
\label{sec:experiments}

Ideally, we would like to be able to show that, with the proper
specifications, dynamics between dominant tax strategies and dominant
auditing policies can be replicated in a computational setting. That
is, we see audit score sheets changing to assign high audit likelihood
to certain transaction sequence behavior that produces relatively low
taxable income. Then in turn, we see the population of transaction
sequences changing to favor transaction sequences that continue to
produce low levels of taxable income, but using techniques that are
not deemed suspicious by the dominant audit score sheets in the
opposing population. We demonstrate STEALTH using a particular known
tax evasion scheme called Installment Bogus Optional Basis~(iBOB).

\subsubsection{iBOB Description}
\label{sec:ibob}

In iBOB, a taxpayer arranges a network of transactions designed to
reduce his tax liability upon the eventual sale of an asset owned by
one of his subsidiaries~\cite{iBOB}.  He does this by stepping up the
basis of this asset according to the rules set forth in \S 755 of the
IRC.  In this way, he manages to eliminate taxable gain while
ostensibly remaining within the bounds of the tax
law~\cite{wright45financial}.

The sequence of transactions, shown graphically in
Figure~\ref{fig:iBob}, for the iBOB scheme are enumerated:

\begin{figure}[htp]
\centering
\begin{subfigure}{.30\textwidth}
  \centering \includegraphics[width=1.0\linewidth]{figures/iBob_1}
  \caption{iBob step 0}
  \label{fig:iBob_1}
\end{subfigure}
\begin{subfigure}{.30\textwidth}
  \centering \includegraphics[width=1.0\linewidth]{figures/iBob_2}
  \caption{iBob step 1}
  \label{fig:iBob_2}
\end{subfigure}
\begin{subfigure}{.30\textwidth}
  \centering \includegraphics[width=1.0\linewidth]{figures/iBob_3}
  \caption{iBob step 2}
  \label{fig:iBob_3}
\end{subfigure}
\caption{The steps in the iBOB tax evasion scheme. The basis of an
  asset is artificially stepped up and tax is avoided by using
  ``pass-through'' entities.}
\label{fig:iBob}
\end{figure}

{\small
\begin{enumerate}
\setcounter{enumi}{-1}
\item In the initial ownership network Mr. Jones is a 99\% partner in
  JonesCo and FamilyTrust, whereas JonesCo is itself a 99\% partner in
  another partnership, NewCo. NewCo owns a hotel with a current fair
  market value~(FMV) of \$200. If NewCo decides to sell the hotel at
  time step 1, Mr. Jones will incur a tax from this sale. The tax that
  Mr. Jones owes is the difference between the FMV at which the hotel
  was sold and his share of inside basis in this hotel,
  i.e. \$199-\$119 = \$80.  Mr. Jones can evade this tax by
  artificially stepping up the inside basis of the hotel to \$199.

\item In the first transaction, we see that FamilyTrust, which
  Mr. Jones controls, decides to buy JonesCo's partnership share in
  NewCo for a promissory note with a current value of \$199.  Of
  course, FamilyTrust has no intention of paying off this note, as any
  such payments entail a tax burden upon NewCo. Having already made a
  754 election, FamilyTrust steps up its inside basis in the hotel to
  \$199.

\item When NewCo sells the Hotel to Mr.Brown for \$200, Mr. Jones does
  not incur any tax, as the difference between the current market
  value and his share of inside basis in the hotel is now zero.
\end{enumerate}
}

In our STEALTH implementation we note in addition to iBOB transaction
(item 1) two additional patterns of transaction activity that can
result in zero immediate tax liability for Mr Jones.  The first of
these involves the transfer of a partnership interest between two
``linked" entities in the same enterprise structure, usually resulting
in a basis adjustment due to an earlier \S 754 election. By ``linked"
we mean a transaction in which the two parties are connected by an
ownership relationship.  In the iBOB context, these include ``singly
linked" transactions, such as those that may occur between Mr Jones
and JonesCo (or JonesCo and NewCo), and ``doubly linked" transactions,
as may occur between Mr Jones and NewCo.  These types of transactions
result in zero immediate tax liability for all parties, but would
almost certainly be audited.  The second such transaction involves the
use of Annuities such as promissory notes that are taxed only at the
time of payment.  As with ``linked" transactions, defaulting on
Annuity payments is nominally legal and results in zero tax liability
but can be very suspicious for auditors.

\subsubsection{Setup}

We ran $100$ independent iterations of the co-evolutionary GA for $100$
generations each with tax scheme and audit score populations of size
$100$. We chose $0.5$ of the tax scheme population for evaluating the
fitness of the solution in the other audit score population and
vice-versa. The parameters that govern the GA simulation are displayed in
Table~\ref{tab:SCOTE_parameters}.

\begin{table}[!htb]
  \small
  \centering
  \caption{Parameters for STEALTH iBOB experiments}
  \label{tab:SCOTE_parameters} 
  \begin{tabular}{l|p{8cm}|l}
    \textit{Parameter} & \textit{Description} & \textit{Value} \\ \hline
    \texttt{Mutation rate}     & probability of integer change in individual  &  $0.1$ \\ \hline
    \texttt{Crossover rate}    & probability of combining two individual integer strings  &  $0.7$ \\ \hline
    \texttt{Tournament Size} & number of competitors when selecting individuals  &  $ 2 $ \\ \hline
    \texttt{Number chosen}   & fraction of other population each individual is evaluated  &  $ 0.5$ \\ \hline
    \texttt{Population size}  & number of individuals in each population &  100   \\ \hline
    \texttt{Generations}     & number of times populations are evaluated  &  100    \\ 
  \end{tabular}
\end{table}

We are doing an initial exploration of the problem and the choice of
parameters and operators are a first attempt. Each population has the
same operator and parameter settings. In the initialization integers
are randomly chosen. Individuals are selected from the population
using tournament selection. In the crossover operation two individuals
are combined into two new individual by randomly picking a single
point and swap the after the point. The grammar used to map the
integers in an individual is shown in Figure~\ref{fig:ge_ex} and the
initial configuration is always the same. The mutation operation of an
individual chooses a new random integer at a random position. The
fitness of an individual is the average over a number of randomly
chosen individuals from the other population.

The objectives functions used for these experiments were single
objective and opposites of one another. Given that Jones's taxable
income was the only one of interest, $\zeta = 1$, and we can define
his taxable income as just $\ell$. Thus the objective function for a
transaction sequence is $h_e(\ell, s) = -\ell(1-s)$. Conversely, the
objective function for audit score sheets are $h_a(\ell, s) = -
\ell(1-s)$. A primary assumption underlying our model is that tax
schemes and audit scores sheets are engaged in a perpetual
co-evolutionary process in which no global attractor exists. To
generate sustaining oscillations, we constrain the resources of the
auditor. We restrain the audit score sheets to assign the lowest audit
point a value of zero, so that there will always be at least one
scheme that is not detectable by the auditor. For the experiments
considered the audit observables are:
\begin{inparaenum}
\item \texttt{Material for Annuity}
\item \texttt{Single Linked}
\item \texttt{Double Linked}
\item \texttt{iBOB}.
\end{inparaenum}
The value of each audit point can be thought of as the relative
importance of the associated transaction to the auditor.

\subsubsection{Coevolution of Auditors \& Evaders in iBOB}
% NOTE fix the figure legends, audit is NOT just the 754 election

For our analysis we pick one run from our experiments. Fitnesses from
various subpopulations of the transaction sequence population are
shown in Figure~\ref{fig:transaction_lag}. A sharp increase in the
fitness of the ``best'' transaction sequence indicated the discovery
of new way to minimize the payment of taxes. As soon as that occurs,
the fitness of the best $10\%$ of transaction sequences increases to
the maximum, shortly followed by the fitness of all of the sequences
in the population. This seems to agree with the real-life observations
of abusive tax shelters, where tax-minimizing schemes quickly
propagate amongst the industry once
discovered~\cite{wright45financial, confidencegames}. Also encouraging
is the combined decline amongst all subpopulation fitnesses,
indicative of the evolution of an audit score (not shown) sheet that
increases the audit likelihood of a transaction sequence exhibiting
the previously discovered scheme

Figure~\ref{fig:best_audit_weights} below shows a nuanced picture of
the audit score sheet population's response to the general trend in
the transaction sequence population. The colored background shows the
audit weight distribution of the most fit audit score sheet in
the population. Conversely, colored the lines show the proportion of
the transaction sequence population that uses the scheme of the
corresponding color. Thus we can see how the proportion of certain tax
schemes follow the existence of the highest fitness audit score sheet.

We observe that an audit score sheet capable of sufficiently auditing
a certain type of tax scheme can co-exist with that scheme for some
time until the frequency of that tax strategy starts to decline. This
demonstrates
\begin{inparaenum}[\itshape a)]
	\item the successful audit score sheet taking time to
          propagate amongst its population and
	\item the a notion of the fitness landscape of the transaction
          sequences.
\end{inparaenum}
That is, audit score sheets have a fitness landscape that allows
successful auditing policies to disseminate slowly. Conversely,
dominant tax-minimization strategies have a more stochastic discovery
process, but successful schemes propagate rapidly once found.

\begin{figure}[!htb]
\centering
\begin{subfigure}{.49\textwidth}
\includegraphics[width=0.99\textwidth]{figures/transaction_fitness_lag_ed_bf}
\caption{{\small Fitness of best transaction sequences (red), mean of
    top ten sequences (green) and mean of population (blue). The dots
    signify points at which a novel tax-minimizing strategy is
    evolved.}}
\label{fig:transaction_lag}
\end{subfigure}
\begin{subfigure}{.49\textwidth}
\centering
\includegraphics[width=0.99\textwidth]{figures/UM_ndist_audit_observations_line_and_best_auditor_weights_bar_ed}
\caption{{\small Audit weights of best audit score sheet and
    proportions of various transaction sequence scheme types in
    population}}
\label{fig:best_audit_weights}
\end{subfigure}
\caption{{\small Example from on co-evolutionary iBOB run.}}
\label{fig:best_solution_example}
\end{figure}

Thus the dynamics we set out to replicate with our model were
achieved. Successful transaction sequences are those that generate low
levels of taxable income for Jones, as well as exhibiting behavior
that is not adequately represented in the audit score sheet
population. Soon enough, the objective functions of the auditing
policies begin to associated that behavior with low taxable income
relative to other transaction sequences that accomplish the same
economic purpose and assign an audit weight to that behavior. The
effectiveness of that tax strategy then decreases until a new
tax-minimizing strategy is found which once again evades all (or most)
existing auditing policies. That strategy then rapidly spreads amongst
the transaction sequence population and the process continues.

Table \ref{tab:best_transactions} show how the best transaction
sequences change over generations for a run. The initial best
transaction sequence is not very large reduction in tax liability
$(1)$, due to the random intitalization of the populations. After two
generations an improved sequence is found $(2)$, although both a
\texttt{Material for Annuity} and \texttt{Double Link} are observed in
it. At generation seven $(3)$ the best transaction sequence can only
be audited with \texttt{Material for Annuity}. By generation 31 $(4)$
the best transaction sequence can be observed by a \texttt{Single
  Link} when Brown buys NewCo from JonesCo and then with the final
transaction Brown buys the hotel from NewCo, which is observed as a
single linked transaction. At generation 33 the best transaction $(5)$
is possible to observe with a \texttt{Material for Annuity}. By
generation 70 $(6)$ the best transaction sequence is the same as at
generation 31 $(4)$ and can be observed by a \texttt{Single Link} when
Brown buys the hotel from NewCo, the partnership he previously
bough. Finally at generation 78 $(7)$ the audit observable
\texttt{iBOB} can capture the best transaction. We observe how all the
audit observables were evaded due to the resource constraints on the
auditor. It was also possible to note properties describe for
co-evolutionary algorithms, namely how the evasion scheme $(4)$ cycled
in the population.

\begin{table}[htb]
  \footnotesize
  \centering
  \caption{Best transactions sequences, the generation it entered the
    population and the audit observables required to detect
    it. Invalid transactions are cleared from the solutions for
    readability.}
  \label{tab:best_transactions}
  \begin{tabular}{p{0.5cm} p{0.8cm}|p{5.2cm}|p{4.0cm}}
    \textbf{Row} & \textbf{Gen} & \textbf{Transactions} & \textbf{Audit observables} \\
    \hline %S1
    $(1)$ & $<2$ &
    Transaction(Brown, JonesCo, Annuity(200, 30), PartnershipAsset.(99, NewCo)); & \\
    & & Transaction(NewCo, Jones, Material(200, Hotel, 1), PartnershipAsset(99, JonesCo)
    &
    None (Not a good sequence)
\\
\hline %S2
$(2)$ & $ <7$ &
Transaction(NewCo, Jones, Material(200, Hotel, 1), Annuity(300, 30))
&
\texttt{Material for annuity} and \texttt{Double linked}
\\
\hline %S3
$(3)$ & $ <31$ &
Transaction(NewCo, FamilyTrust, Material(200, Hotel, 1), Annuity(300, 30))
&
\texttt{Material for annuity}
 \\
 \hline %S4
 $(4)$ & $ <33$ &
 Transaction(Brown, JonesCo, Annuity(200, 30), PartnershipAsset(99, NewCo))
 &
\texttt{Single linked} (Brown buys hotel from himself)
 \\
 \hline %S5
 $(5)$ & $ <70$ &
 Transaction(Brown, JonesCo, Annuity(200, 30), PartnershipAsset.(99, NewCo)); & \\
 & & Transaction(NewCo, Jones, Material(200, Hotel, 1), Annuity(300, 30))
 &
\texttt{Material for annuity}
 \\
 \hline %S6
 $(6)$ & $ <78$ &
 Transaction(Brown, JonesCo, Annuity(200, 30), PartnershipAsset(99, NewCo))
 &
 \texttt{Single linked} (same as at $(4)$)
 \\
 \hline %S7
 $(7)$ & $ <100$ &
 Transaction(FamilyTrust, JonesCo, Annuity(200, 30), PartnershipAsset(99, NewCo))
 &
 \texttt{iBOB}
 \\
  \end{tabular}
\end{table}

There are calibrations that can improve the fidelity of the
experiments. For example, while transaction sequences are clearly more
responsive to a successful individual in their population than audit
score sheets, the time scale gives too much credit to the propagation
of audit score sheets. For example, Figure~\ref{fig:transaction_lag}
shows that a successful tax strategy enjoys only about $5-10$
generations of unbridled prosperity until an auditing policy evolves
and propagates that reduces its effectiveness. Transaction sequences
take about the same amount of generations to figure out a new dominant
tax strategy, the only tangible difference is the speed at which it
propagates through the population. Thus, there must be further
calibration in the model to reflect the differences in time scale.
