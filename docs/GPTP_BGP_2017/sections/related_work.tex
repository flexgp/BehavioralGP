\section{Related Work}
\label{sec:related-work}

BGP is among a number of other approaches to program synthesis where progress has recently become more empirically driven, rather than driven by formal specifications and verification\cite{Basin04synthesisof}.
Alternate approaches to evolutionary algorithms include e.g. sketching\cite{solar2008program}, i.e. communicating insight through a partial program, generalized program verification\cite{srivastava2010program}, as well as hybrid computing with neural network and external memory\cite{graves2016hybrid}

BGP takes inspiration, with respect to its focus on program behavior, from earlier work on implicit fitness
sharing\cite{mckay2000fitness}, trace consistency analysis and the use of archives as a form of memory\cite{haynes1997line}. 
In its introduction in \cite{krawiec2013pattern} the preceding introduction of
the trace matrix was noted within a system called Pattern Guided Genetic Programming, ``PANGEA''. 
In PANGEA  minimum description length was induced from the program execution.
Subsequently there have been a variety of extensions. For example, in the general
vein of behaviorally characterizing a program by more than its
accuracy \cite{liskowski2016online} considers discovering search objectives for test-based
problems. Also notable is Memetic Semantic Genetic
Programming\cite{Ffrancon:2015:MSG:2739480.2754697}.

BGP was introduced as a broader and more detailed take on Semantic GP. BGP and Semantic GP share the common goal of
characterising program behaviour in more detail and exploiting this
information. Whereas BGP looks at subprograms, semantic GP focuses on
program output.  Output is scrutinized for every test individually and a study of the impact of crossover on program semantics
and semantic building blocks~\cite{mcphee2008semantic} was conducted. A survey of
semantic models in GP~\cite{vanneschi2014survey} gives an overview of
methods for their operators, as well as different objectives.

