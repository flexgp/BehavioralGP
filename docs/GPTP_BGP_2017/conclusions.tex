\section{Conclusions \& Future Work}
\label{sec:conclusions--future}

We presented how Genetic Programming, grammars and coevolutionary
algorithms could be used in practice. We expanded on the role of
dynamic environments and coevolution in GE. We used a coevolutionary
algorithm to replicate the co-evolutionary relationship between tax
evaders and auditors, using US partnership taxation as an initial
example. We proceeded with
\begin{inparaenum}[\itshape (1)]
\item representing the rule system in order to calculate benefit that
  the advisor can offer to their client
\item simulating interactions between the advisor's strategy and the
  relevant regulatory authority, and
\item optimizing for behavior on both ends of the relationship to
  investigate potential areas of exploration.
\end{inparaenum}

The co-evolutionary relationship can be replicated through
experimentation, given the proper specifications. Some further
parameter calibrations are required in order to capture certain time
scale effects, but the qualitative dynamics are present. Transaction
sequences can be shown to respond to both tax minimizing behavior and
risk of being audited. Similarly, auditing policies respond to and
isolate behavior which generates lower than expected taxable income.

For future work we will expand our representation of US partnership
tax code we would like to non-recourse liabilities and depreciation
deduction schedules as a means to minimize taxable income. Another key
aspect of validation is to gain access to actual auditing data. This
is a non-trivial process that requires security clerance clearance.
Additional future work is to analyzing the coevolutionary algorithm
and dynamics, i.e. how the solutions in the populations are coevolving
during the co-evolutionary search. The operators used for the search,
pareto-archives, cycling of solutions and multi-objective fitness.
