\section{Related work}
\label{sec:related-work}

In the STEALTH framework for anticipating tax evasion we have used
previous work in GE and coevolutionary algorithms. In this section we
describe the Genetic Programming approach with grammars called
Grammatical Evolution in Section~\ref{sec:gramm-evol}. Coevolutionary
algorithms are in Section~\ref{sec:co-evolution}. Finally, the work on
coevolution and GE are in Section~\ref{sec:co-evol-gramm}

\subsection{Grammatical Evolution}
\label{sec:gramm-evol}

Grammatical Evolution~(GE) is a version of genetic programming with a
variable length integer representation and an indirect mapping using a
grammar~\citep{o2003grammatical}, which allows us to generate valid
phenotypes, which in our case are transaction sequences and audit
policies. As shown in Figure~\ref{fig:ge_ex}, the grammar is composed
of a single start symbol, terminal symbols and non-terminal symbols,
as indicated by the boxes in the figure. The left hand box shows the
grammar in BNF form and the right hand box shows the rewriting of
integers to terminal symbols and the corresponding derivation tree. In
the rewriting the start symbol is at the top and terminal symbols are
at the bottom of each branch of the derivation tree. Integers are fed
into the top and the direction of the path is determined by taking the
modulo of the current integer, at which point the next integer is
selected. The process is complete when the sentence comprises only
terminal symbols.

\begin{figure*}[!htb]
\centering \includegraphics[width=0.99\textwidth]{figures/ge_ex}
\caption{{\small Example of how GE rewrites a list of integers
    (Genotype) into a list of transactions (Phenotype) with a BNF
    grammar}}
\label{fig:ge_ex}
\end{figure*}

The use of a grammar has allowed GE to be applied to many different
areas. For example, games, finance, design, hyper-heuristic for
combinatorial optimization problems and parallel sorting programs on
multi-cores~\citep{shaker2012evolving, dempsey2009foundations,
  byrne2014evolving, sabar2013grammatical,
  chennupati2015automatic}. GE provides a division of search and
solution space. The indirect encoding of the grammar allows search
space transformations and both constrain and bias the search. In a
proof-of-concept application, as currently for STEALTH, the usability
and incorporation of domain knowledge currently outweighs the impact
of GE operators, e.g. low locality~\citep{thorhauer2014locality,
  whigham2015examining}.

\subsection{Co-evolution}
\label{sec:co-evolution}

In biology, co-evolution describes situations where two or more
species reciprocally affect each other's evolution. The notion of
adversarial co-evolution from biology can be used for the
circumstances of the auditors, e.g. each time the IRS changes the tax
code the tax evaders react by finding new ambiguities. The auditor and
the tax evaders are \textit{co-evolving} as interacting species. At
its core, the overall dynamics of the system reflect a constantly
transitioning series of complementary adjustments, with each
predator/prey seeking to bring advantage to the predator/prey under
adjustment.

Coevolutionary algorithms explore domains in which no single
evaluation function is present or known. Instead, algorithms rely on
the aggregation of outcomes from interactions among coevolving
entities to make selection decisions. The lack of an explicit
measurement, understanding the dynamics of coevolutionary algorithms
and determining the progress of a given algorithm present further
challenges~\citep{popovici2012coevolutionary}. Usually, Evolutionary
Algorithms begin with a fitness function, which for the purposes of
this chapter is a function of the form $f: G \mapsto \mathbb{R}$ that
assigns a real value to each possible genotype in $G$. Individual
solutions are compared as $f(g_0)$ with $f(g_1)$ and the fitness
ranking is always the same. In coevolution two individuals are
compared based on their outcome from interaction with other
individuals, thus the ranking of an individual solution can change
over time.

Coevolution is appropriate for domains that have no intrinsic
objective measure, also called \textit{interactive} domains. There are
two types of coevolution:
\begin{inparadesc}
\item [Compositional] for a problem the quality of a solution to the
  problem involves an interaction among many components that together
  might be thought of as a team, i.e. cooperating.
\item [Test-based] for a problem is one in which the quality of a
  potential solution is determined by its performance when interacting
  with some set of tests, i.e. competitive. E.g. the interactive test
  for a tax evasion strategy are different auditor behaviors, and vice
  versa.
\end{inparadesc}

Coevolutionary algorithms and game theory are related which leads to
applications in games~\citep{popovici2012coevolutionary}. A
distinction from game theory is that co-evolutionary EAs can be
applied to larger search spaces~\citep{rush2015coevolutionary}.  Some
work has investigated solution concepts for test-case coevolution with
a no-free-lunch framework\citep{tauritz2008no}. In other applications,
some of the problems with streaming data classification are
encountered in coevolution~\citep{heywood2015evolutionary}. Moreover,
co-evolution has been compared to boosting~\citep{ibabagging}. In
addition, co-evolution is also used for complexification of
solutions~\citep{stanley2004competitive}. Finally, there are also
applications to simulations of behavior, e.g. Zero-Day exploit
strategies in Cyber Security~\citep{winterrose2014strategic} and the
search for bug fixes~\citep{le2013moving}, where the program already
exists. STEALTH addresses the EA ``middle''
strategies~\citep{rush2015coevolutionary}.

\subsection{Co-evolution and Grammatical Evolution}
\label{sec:co-evol-gramm}

GE has been applied with coevolutionary algorithms. In the GE
literature on dynamic environments~\citep{dempsey2009foundations}
co-evolution is characterized as ``Markov'', i.e. it has memory and
proceeds from its current state. In coevolution the fitness of
solutions are dependent on the search process, this dependency does
not have to hold for dynamic environments. Some coevolutionary
literature view interactive domains as static, with a different
structure. In Grammatical evolution by grammatical
evolution~(GEGE)~\citep{dempsey2009foundations} tries to
simultaneously evolve the grammar and genetic code with a hierarchy of
grammars. The aim is to find ``modules'' that can be reused when the
environment changes, i.e. in a dynamic environment. GEGE is a compact
representation of a larger grammar with an increased search space and
strongly couples the grammars~\citep{azad2006examination}.

Examples of applications of GE and coevolution are for reformulation
of training as a two-population competition, that is learners versus
training exemplars and use GE to represent Pareto-coevolutionary
classifier and Multi-objective classifier~\citep{mcintyre2007multi,
  mcintyre2008cooperative}.  An Artificial Life model for evolving a
predator--prey ecosystem of mathematical expressions with
GE~\citep{alfonseca2015evolving}. Coevolutionary algorithms with GE
for financial trading, e.g. using multiple cooperative
populations~\citep{adamu2011coevolutionary,
  gabrielsson2014co}. Spatial co-evolution in age layered planes
\citep{harper2014evolving} for robots in robocode using competitive
coevolution.

STEALTH requires the tax regulations to be encoded in software before
the search for strategies can be performed. The tax law is defined by
existing rules, in the IRC. The solutions in the separate populations
in STEALTH are both interesting. Next we describes how to use GE and
multi population interactive test based coevolution with US
partnership taxation.
